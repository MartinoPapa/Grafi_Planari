%prerequisiti

\begin{definizione}[Grafo]
    Un grafo è una coppia \(G=(V,E)\) dove:
    \begin{itemize}
        \item \(V\) è un insieme di vertici (\textbf{vertex});
        \item \(E\) è un insieme di coppie di nodi \((u,v),\;u,v\in V\) dette archi o lati (\textbf{edge});
    \end{itemize}
    Nei grafi \textbf{orientati} le coppie in \(E\) sono ordinate in quelli non orientati no.
\end{definizione}

\begin{definizione}[Adiacenza]
Un vertice \(v\) si dice adiacente a \(u\) se esiste \((u,v) \in E\). \\
NB:\@ nei grafi non orientati l'adiacenza è una relazione simmetrica.
\end{definizione}

\begin{definizione}[Incidente]
    Un arco \((u,v)\) si dice incidente da \(u\) a \(v\).
\end{definizione}

\begin{definizione}[Grado]
    Nel caso di grafi orientati definiamo:
    \begin{itemize}
        \item \textbf{grado entrante} di un nodo come il numero di archi incidenti su esso;
        \item \textbf{grado uscente} di un nodo come il numero di archi incidenti da esso.
    \end{itemize}
    Per i grafi non orientati avremo invece un'unica definizione:
    \begin{itemize}
        \item il \textbf{grado} di un nodo è il numero di archi incidenti su di esso. 
    \end{itemize}
\end{definizione}

\begin{definizione}[Cammino]
    Sia \(G=(V,E)\) un grafo. Un cammino \(C\) di lunghezza \(k\) è una sequenza di nodi \(u_0,u_1 \dots, u_k\) t.c. 
    \begin{equation}
        (u_i,u_{i+1})\in E \text{ per } 0 \leq i \leq k-1
    \end{equation}
\end{definizione}

\begin{definizione}[Grafi isomorfi]
    Siano \(G=(V(G),E(G)), H=(V(H),E(H))\) due grafi. Diremo \(G\) isomorfo a \(H\) se \(\exists \theta : V(G) \to V(H)\) t.c. \(\theta\) è un isomorfismo e
    \begin{equation}
        \theta(E(G)) \doteq \{ \theta(uv) \; t.c.\; uv \in E(G)\} = E(H)
    \end{equation}
\end{definizione}

\begin{definizione}[Grafo completo]
    Un grafo \(G=(V,E)\) si dice completo se
    \begin{equation}
        \forall u,v \in V \; \exists (u,v) \in E
    \end{equation} 
    Definiamo \(K_n\) un grafo completo con \(n\) vertici.
\end{definizione}

\begin{definizione}[Grafo bipartito]
    Un grafo non orientato \(G=(V,E)\) si dice bipartito se \(V\) può essere diviso in due sottoinsiemi \(X,Y\) t.c.
    \begin{equation}
        \forall (u,v) \in E \text{ vale } u\in X,\; v\in Y \text{ oppure } u\in Y,\; v\in X  
    \end{equation}
    Un grafo bipartito può avere al più \(|X|\cdot |Y|\) archi. \\
    Definaiamo inoltre \(K_{m,n}\) il grafo bipartito completo che soddisfa
    \begin{equation}
        |X|=m,\; |Y|=n,\; \varepsilon \doteq |E| = mn
    \end{equation}
\end{definizione}


\begin{definizione}[Connessione]
    Un grafo non orientato \(G=(V,E)\) è detto \textbf{connesso} se
    \begin{equation}
        \forall u,v \in V \; \exists (u,v) \in E
    \end{equation}
    Un sottografo connesso massimale di un grafo non orientato è detto \textbf{componente connessa}.
\end{definizione}

\begin{definizione}[Vertex-connectivity]
    Sia \(G=(V,E)\) un grafo non orientato. Definiamo \textbf{vertex-connectivity}  \(\kappa\) il minimo numero di vertici da eliminare per sconnettere \(G\).
\end{definizione}

\begin{definizione}[Grafo k-connesso]
    Sia \(G=(V,E)\) un grafo non orientato. Diremo \(G\) \textbf{k-connesso} se \(|V|>k\) e \(\kappa \geq k\) dove \(\kappa\) corrisponde alla vertex-connectivity.\\
    Informalmente un grafo è detto k-connesso se rimane connesso rimuovendo \(k'<k\) vertici qualsiasi.
\end{definizione}

\begin{teorema}[Grafo 2-connesso]
    Un grafo non orientato \(G=(V,E),\; |V| \geq 3\) è 2-connesso \(\Leftrightarrow\) ogni coppia di vertici \((u,v)\) è connessa da almeno 2 cammini internamente disgiunti.
\end{teorema}

\begin{definizione}[Cammini internamente disgiunti]
    Sia \(G=(V,E)\) un grafo non orientato, siano \(a,b \in V\) due cammini da \(a\) a \(b\) \(a,v_1,\dots,v_n,b\), \(a,u_1,\dots,u_n,b\). Essi si dicono internamente disgiunti se 
    \begin{equation}
        v_i \neq u_j\; \forall i,j
    \end{equation}
\end{definizione}

\begin{definizione}[Ciclo di Hamilton]
    Sia \(G=(V,E)\) un grafo non orientato. Definiamo ciclo di Hamilton un ciclo che contiene tutti i vertici del grafo.\\
    Definiamo \(G\) hamiltoniano se \(G\) contiene un ciclo di hamilton.
\end{definizione}
% teo

\begin{definizione}[Grafo planare]
    Un grafo non orientato \(G\) si dice planare se può essere rappresentato nel piano evitando che gli archi si intersechino (se non negli endpoint).
\end{definizione}

\begin{teorema}[Formula di Eulero]\label{formulaeulero}
    Sia \(G\) un grafo planare connesso con \(V\) vertici e \(\epsilon\) lati. Sia \(G^\varphi\) una immersione di \(G\) nel piano avente \(f\) faccie. Allora
    \begin{equation}
        V+f-\epsilon = 2
    \end{equation}
\end{teorema}

\begin{definizione}[Suddivisione]
    Dato un grafo \(G\) definiamo suddivisione (subdivision) di \(G\) i grafi ottenuti da \(G\) rimpiazzando uno o più archi con cammini di lunghezza 2 o più.
    \\ Definiamo inoltre contrazione l'operazione inversa della suddivisione.
\end{definizione}
\begin{lemma}
    Ogni suddivisione di un grafo non planare è non planare.
\end{lemma}

\begin{definizione}[Grafi omeomorfi]
    Due grafi \(G_1,G_2\) si dicono topologicamente equivalenti o omeomorfi se possono essere trasformati l'uno nell'altro attraverso operazioni di suddivisione o contrazione degli archi.
    \\ Denotiamo l'insieme dei grafi omeomorfi a \(G\) con \(TG\).
\end{definizione}

\begin{definizione}[Minore]
    Sia \(H\) un grafo ottenuto da \(G\) tramite una sequenza di operazioni di rimozione di archi o vertici o contrazione di archi. \(H\) è detto minore di \(G\).
\end{definizione}

\begin{teorema}[Teorema di Kuratowski]
    Un grafo \(G\) è planare se e solo se non contiene sottografi \(TK_{3,3}\) o \(TK_5\).
\end{teorema}
\noindent Si può riformulare il teorema precedente in termine di minori.
\begin{teorema}[Teorema di Wagner]
    Un grafo \(G\) è planare se e solo se non ha \(K_{3,3}\) o \(K_5\) come minori.
\end{teorema}

% Teoria per algoritmo di Hopcroft Trajan
\subsubsection{Teoria usata direttamente nell'algoritmo Hopcroft Trajan}

\begin{lemma}
    Sia \(G\) un grafo planare, \(G^\psi\) una sua immersione nel piano, \(F_1, \dots, F_f\) le faccie di \(G^\psi\) allora
    \begin{equation} \label{gradifaccie}
        \sum_{i=1}^f \deg(F_i) = 2\epsilon
    \end{equation}
\end{lemma}
\begin{proof}
    Segue direttamente dal fatto che ogni arco \((u,v)\) è incedente esattamente su due faccie di \(G^\psi\).
\end{proof}

\begin{lemma}
    Sia \(G\) planare, \(\epsilon\) il numero di archi, \(V\) il numero di vertici. Vale allora
    \begin{equation}
        \epsilon \leq 3V - 3
    \end{equation}
    Inoltre se supponiamo \(V \geq 3\) vale
    \begin{equation}
        \epsilon \leq 3V - 6
    \end{equation}
\end{lemma}
\begin{proof}
    Sia \(V<3\). In questo caso il lemma è una diretta conseguenza della formula di Eulero (\ref{formulaeulero}). \\
    Sia \(V\geq 3\). \(G\) planare \(\Rightarrow\) ogni faccia ha almeno 3 lati ovvero \(\deg(F_i) \geq 3\; \forall i\). Per il lemma (\ref{gradifaccie}) vale quindi
    \begin{equation}
        2\epsilon = \sum_{i=1}^f \deg(F_i) \geq \sum_{i=1}^f 3 = 3f
    \end{equation}
    Per la formula di eulero segue inoltre \(V+f-\epsilon = 2\) da cui abbiamo, moltiplicando per 3 e riarrangiando i termini
    \begin{equation}
        3\epsilon = 3V + 3f - 6
    \end{equation}
    applicando ora \(3f\leq 2 \epsilon\) otteniamo la tesi.
\end{proof}
